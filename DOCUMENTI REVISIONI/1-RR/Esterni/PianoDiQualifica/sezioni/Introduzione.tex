\subsection{Scopo del documento}
Lo scopo di tale documento consiste nello stabilire le norme utilizzate per verificare la qualità del prodotto e del processo. A tale scopo sarà svolta una continua attività di verifica per rilevare e correggere eventuali anomalie tramite metriche di seguito descritte.
\subsection{Scopo del prodotto}
Lo scopo del prodotto è quella di realizzare un'applicazione \citgl{Android} che implementi un servizio di car sharing \citgl{peer-to-peer}.
\subsection{Glossario}
Onde evitare ambiguità o incomprensioni di natura lessicale, si allega il \G.
All'interno del documento saranno presenti parole di ambito specifico, uso raro che potrebbero creare incomprensioni. Per una maggiore leggibilità tali parole sono riconoscibili all'interno dei vari documenti in quanto scritte in corsivo e con un 'g' a pedice tra barre orizzontali (per esempio \citgl{Glossario})
\subsection{Riferimenti}
    \subsubsection{Riferimenti normativi}
    \begin{itemize}
        \item \NdP
        \item Capitolato C5: Car sharing peer to peer.
        \\ \url{ https://www.math.unipd.it/~tullio/IS-1/2018/Progetto/C5.pdf}
    \end{itemize}
    \subsubsection{Riferimenti informativi}
    \begin{itemize}
        \item Slide corso Ingegneria del Software:
        \\ \url{https://www.math.unipd.it/~tullio/IS-1/2018/Dispense/L03.pdf}
        \item Slide corso Ingegneria del Software: Qualità di prodotto
        \\ \url{https://www.math.unipd.it/~tullio/IS-1/2018/Dispense/L13.pdf}
        \item Slide corso Ingegneria del Software: Qualità di processo
        \\ \url{https://www.math.unipd.it/~tullio/IS-1/2018/Dispense/L14.pdf}
        \item Standard ISO/IEC 9126
        \\ \url{https://it.wikipedia.org/wiki/ISO/IEC_9126}
        \item Indice Gulpease:
        \\ \url{https://it.wikipedia.org/wiki/Indice_Gulpease}
        \item Framework Octalysis:
        \\ \url{https://yukaichou.com/gamification-examples/}
        \item ISO/IEC 15504:
        \\ \url{https://en.wikipedia.org/wiki/ISO/IEC_15504}
        
    \end{itemize}
    
    