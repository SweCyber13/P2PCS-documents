\subsubsection{Pianificazione}
Per la pianificazione la scelta è ricaduta sulla piattaforma GitHub perchè offre la possibilità di organizzare le task di progetto in ticket, da assegnare a uno o più membri ed infine organizzarli in colonne. A tal proposito il gruppo ha deciso di creare varie colonne una per ogni fase del ciclo di vita del ticket in modo da avere un idea chiara sul punto in cui sono i vari membri del gruppo con i relativi compiti.
\subsubsection{Comunicazione}
Per la comunicazione ci si è orientati per l'applicazione di messaggistica \citgl{Slack} apposita per gruppi di lavoro, che offre la possibilità di realizzare canali telematici, in modo da creare comunicazioni specifiche per argomento.
Per le comunicazioni interne al gruppo finora non si ha avuto la necessità di utilizzare strumenti per video chiamate in quanto ci si è organizzati in modo efficiente sugli incontri settimanali. In caso di necessità futura nei si prevede di utilizzare \citgl{Skype} e \citgl{Google Meet} che permettono video chiamate di gruppo.
\subsubsection{Creazione diagrammi di Gantt}
Per la realizzazione di diagrammi di \citgl{Gantt} si è scelto lo strumento \citgl{open source} e  \citgl{multipiattaforma} GanttProject.

\subsubsection{Calcolo del consuntivo}
Gli strumenti a disposizione dal Responsabile di Progetto sono i seguenti:
\begin{itemize}
    \item Fogli Google
    \item GanttProject
\end{itemize}
\item