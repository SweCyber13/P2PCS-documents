\subsubsection{Scopo}
In  questa  sezione  vengono sancite precise norme per quanto riguarda il rapporto di fornitura con la proponente \citgl{GaiaGo} e  dei committenti  Prof. Tullio Vardanega e Prof. Riccardo Cardin nell’ambito della progettazione, sviluppo e consegna del prodotto P2PCS. \\
Le norme  ivi trattate sono tassative e i membri del gruppo Cyber13  sono tenuti a rispettarle al fine di proporsi e diventare fornitori.

\subsubsection{Aspettative}
Il gruppo intende mantenere un costante dialogo con il proponente per avere un riscontro efficace sul lavoro svolto.\\
Nel corso dell'intero progetto il team Cyber13 si auspica di instaurare con la committente GaiaGo, nel particolare con il referente Filippo Pretto, un rapporto di collaborazione al fine di:
    \begin{itemize}
        \item Determinare come adempire appieno ai requisiti fissati dal proponente;
        \item Determinare vincoli sui processi e i requisiti;
        \item Eseguire una stima dei costi;
        \item Concordare la qualifica del prodotto.
    \end{itemize}


\subsubsection{Attività}
    \paragraph{Studio di fattibilità}
    ~\\
    In seguito alla formazione ufficiale dei gruppi del secondo lotto, avvenuta in data 4 Marzo 2019, è stata convocata una riunione interna al gruppo per discutere in merito ai vari capitolati disponibili per lo svolgimento del progetto, presentati in data 16 Novembre 2018.\\
    Una volta stabilita la scelta del capitolato per il quale proporsi come fornitori, gli analisti hanno condotto un’ulteriore e approfondita attività di analisi dei rischi e delle opportunità culminata con la redazione del documento \SdF .
    Tale documento include le motivazioni che hanno portato il gruppo Cyber13 a
    proporsi come fornitore per il prodotto indicato e riporta per ciascun capitolato:
    \begin{itemize}
        \item \textbf{Informazioni sul capitolato}: In cui vengono riportati nome del progetto, azienda proponente e i committenti.
        \item \textbf{Descrizione}: All'interno della quale viene descritto brevemente cosa il progetto richiede.
        \item \textbf{Studio del dominio}: Nel quale vengono indicati i domini applicativi e tecnologici del capitolato.
        \item \textbf{Esito finale}: Dove sono riportati aspetti positivi, fattori di rischio e conclusione finale del gruppo in merito al capitolato.
    \end{itemize}
    
    \paragraph{Piano di progetto}
    ~\\
    Il responsabile avrà il compito di redigere il documento \PdP. Tale documento formale descrive:
    \begin{itemize}
        \item Gli obiettivi del progetto.
        \item Le attività da svolgere per adempire agli obiettivi rispettando i requisiti fissati. Attraverso l'attività di pianificazione si organizzano le diverse attività da svolgere con precise tempistiche da rispettare.
        \item Le risorse coinvolte (in termini di personale e di tecnologie impiegate).
        \item I costi previsti, che vengono stabiliti attraverso il preventivo e il consuntivo. Sulla base della pianificazione si stima la quantità di lavoro necessaria per ogni fase, proponendo così un preventivo per il costo totale del progetto. Alla fine di ogni attività si redige inoltre un consuntivo di periodo per tracciare l’andamento rispetto a quanto preventivato;
        \item I rischi potenziali attraverso l'analisi dei rischi. Si analizzano nel dettaglio i rischi che potrebbero insorgere nel corso del progetto e i modi per affrontarli, capendo la probabilità che essi accadano e il livello di gravità ad essi associato.
    \end{itemize}
    
    \paragraph{Piano di qualifica}
     I verificatori dovranno scegliere una strategia da adottare per la \citgl{verifica} e la \citgl{validazione} del materiale prodotto dal gruppo. Il documento dovrà contenere:
     \begin{itemize}
         \item \textbf{Visione generale delle strategie di verifica}: Si stabiliscono le procedure di controllo sulla qualità di processo e di prodotto, tenendo in considerazione le risorse a disposizione.
         \item \textbf{Misure e metriche}: Si devono stabilire delle metriche oggettive per i documenti, i processi e il software.
         \item \textbf{Gestione della revisione}: Si devono stabilire le modalità di comunicazione delle anomalie e le procedure di controllo per la qualità di processo.
         \item \textbf{Pianificazione del collaudo}: Si devono definire nel dettaglio le metodologie di collaudo del prodotto realizzato.
         \item \textbf{Resoconto delle attività di verifica}: Alla fine di ogni attività si devono riportare le metriche calcolate e un resoconto sulla verifica di tale attività.
     \end{itemize}














