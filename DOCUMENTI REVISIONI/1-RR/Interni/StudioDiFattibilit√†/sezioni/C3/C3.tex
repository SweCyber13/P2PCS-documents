\subsection{Informazioni sul capitolato}
    \begin{itemize}
        \item Nome del progetto: G\&B - monitoraggio intelligente di processi DevOps
        \item \textbf{Proponente}: Zucchetti
        \item \textbf{Committente}: Prof. Tullio Vardanega, Prof. Riccardo Cardin
    \end{itemize}
\subsection{Descrizione}
L'azienda che propone il capitolato presenta un prodotto \citgl{open source}, Grafana, con cui monitora i propri sistemi e da la possibilità di creare \citgl{plug-in} per personalizzarne l'uso. Il capitolato infatti richiede lo sviluppo di quest'ultimo che grazie alle \citgl{reti Bayesiane} ha lo scopo di prevedere quali potrebbero essere gli interventi da eseguire e le zone di intervento nella linea di produzione del software. La scelta dell'utilizzo delle \citgl{reti Bayesiane} è data dall'esigenza di raccogliere la competenza degli esperti in un sistema probabilistico, collegata poi ai dati effettivamente raccolti sul campo per determinare infine quali eventi non ancora presentatosi saranno più probabili. 

\subsection{Studio del dominio}
     \begin{itemize}
        \item \textbf{Dominio applicativo}: Nel dettaglio l'azienda ha lo scopo di utilizzare tale servizio per monitorare allarmi e segnalazioni tra gli operatori del servizio \citgl{Cloud} e la linea di produzione del software.
        \item \textbf{Dominio tecnologico}:
            \begin{itemize}
                \item Grafana: conoscenza dell'\citgl{ambiente di sviluppo} del \citgl{plug-in};
                \item Javascript: linguaggio su cui si basa il plug-in;
                \item \citgl{Reti Bayesiane} come modello di gestione dei dati.
            \end{itemize}
    \end{itemize}
\subsection{Esito finale}
    \begin{itemize}
        \item Aspetti positivi: Il progetto si presenta molto interessante dal punto di vista degli ambiti d'uso e dell'ambiente di sviluppo, la presentazione risulta molto dettagliata quindi si ha la chiara visione del lavoro da svolgere.
        \item Fattori di rischio:
            \begin{itemize}
                \item La documentazione dell'ambiente di sviluppo Grafana risulta poco soddisfacente per garantire un apprendimento sufficiente alla creazione dell'applicazione richiesta;
                \item Il capitolato non è più disponibile.
            \end{itemize}
        \item Conclusioni:
            \begin{itemize}
                \item Sebbene il capitolato sia stato giudicato complessivamente molto interessante dai componenti del gruppo, esso non è più disponibile;
		        \item Scelta: rigettato.
            \end{itemize}
    \end{itemize}