\subsection{Informazioni sul capitolato}
    \begin{itemize}
        \item Nome del progetto: Soldino
        \item \textbf{Proponente}: Red Babel
        \item \textbf{Committente}: Prof Tullio Vardanega, Prof. Riccardo Cardin
    \end{itemize}
\subsection{Descrizione}
Il capitolato ha come scopo lo sviluppo di diverse \citgl{D-apps} (ovvero applicazioni che usano contratti intelligenti per il loro trattamento)
in esecuzione sulle EVM (Ethereum Virtual Machine). La piattaforma desiderata avrà tre attori principali, ognuno dei quali con funzionalità ben definite: 
\begin{itemize}
        \item Il governo: responsabile della coniazione e distribuzione di denaro ai cittadini e alle imprese. All'interno del sistema la valuta usata è un \citgl{token} compatibile con \citgl{ERC20}, detta Cubit. Inoltre detiene una lista di attività di tutte le aziende autorizzate ad usufruire del sistema;
	   
        \item Gli imprenditori: Le cui attività possono comprare e/o fornire 
        beni o servizi.
	    Inoltre l'azienda su base trimestrale è obbligata a pagare una tassa, IVA, al governo;

        \item I cittadini: acquistano beni e servizi dalle aziende attraverso i Cubit. Ed eventualmente se interessati possono aprire la propria attività registrandosi all'opportuno registro detenuto dal governo.
       
    \end{itemize}
L'interazione tra le varie componenti avverrà tramite un insieme di contratti intelligenti.
Pertanto il progetto sarà composto da due macro moduli:
    \begin{itemize}
        \item WEB/UI: deve contenere un insieme di pagine Web che consentano all'utente di interfacciarsi con la EVM, e quindi di eseguire le operazioni che gli sono consentite;
    
        \item Smart contract: contengono le transazioni che vengono sviluppate con un meccanismo di deposito in garanzia dalla rete Ethereum. Le transazioni vengono memorizzate in una struttura dati detta \citgl{Blockchain}, che sostanzialmente è una catena di blocchi contenenti appunto le transazioni. 
    \end{itemize}
La validazione delle singole transazioni è affidata a un meccanismo di consenso distribuito su tutti i nodi della rete Ethereum.


\subsection{Studio del dominio}
     \begin{itemize}
        \item \tetxbf{Dominio applicativo}: Il capitolato fa riferimento al contesto dell' \citgl{e-commerce} e pagamento di tasse tramite criptovaluta;
        \item \textbf{Dominio tecnologico}:
            \begin{itemize}
                \item EVM (Ethereum Virtual Machine): Piattaforma che permette la gestione della rete Ethereum, lo stato interno e il calcolo. Consente quindi 
                \item Ethereum: Piattaforma per consentire agli utenti di scrivere \citgl{D-apps} che utilizzano tecnologia \citgl{blockchain};
		        scritti in linguaggio Solidity;
		        verificare e eseguire codice sulla \citgl{blockchain}. Il codice in esecuzione sulla EVM è contenuto nei cosiddetti "contratti intelligenti", 
                \item Framework Truffle: \citgl{Framework} di sviluppo per Ethereum che ne facilita lo sviluppo. Ad esso sono delegati la maggior parte dei compiti di
		        routine.
                \item Meta Mask: \citgl{Plug-in} browser che consente di eseguire le dApp di Ethereum usando un nodo pubblico invece che eseguire un nodo Ethereum completo.
		        Include inoltre un vault di identità protetto che fornisce all'utente l'interfaccia per gestire la sua identità su diversi siti e firmare
                 \item Ropsten: Rete di test ufficiale, creata da The Ethereum Foundation;
		        transazioni Ethereum in \citgl{blockchain}; 
		         a basso costo e scalabili sulla rete Ethereum. Funziona con qualsiasi token compatibile a ERC20.
                 \item Raiden Network: Strato infrastrutturale sopra la blockchain di Ethereum. Ha lo scopo di consentire pagamenti quasi istantanei,
            \end{itemize}
    \end{itemize}
\subsection{Esito finale}
    \begin{itemize}
        \item Aspetti positivi:
             \begin{itemize}
                \item Studio della piattaforma Ethereum, reputata molto interessante dai componenti del gruppo;
                \item Studio del linguaggio orientato agli oggetti Solidity per la creazione di Smart Contracts;
                \item Il framework Truffle facilita lo sviluppo, in quanto facilita e automatizza tutte le operazioni di routine;
            \end{itemize}
        \item Fattori di rischio:
            \begin{itemize}
                \item La maggior parte delle tecnologie è completamente sconosciuta ai componenti del gruppo, e lo studio di un numero di tecnologie così alto
		        potrebbe non essere sostenibile dal gruppo.
            \end{itemize}
        \item Conclusioni
            \begin{itemize}
                \item Sebbene il capitolato sia stato giudicato complessivamente molto interessante dai componenti del gruppo, il tempo necessario per lo
		        studio dei framework utilizzati da Soldino è stato considerato non sostenibile;
		        \item Scelta: rigettato.
            \end{itemize}
    \end{itemize}